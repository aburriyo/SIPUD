\documentclass[11pt,a4paper]{book}
\usepackage[spanish]{babel}
\usepackage[utf8]{inputenc}
\usepackage[T1]{fontenc}
\usepackage{xcolor}
\usepackage{hyperref}
\usepackage{geometry}
\usepackage{fancyhdr}
\usepackage{tocloft}
\usepackage{enumitem}
\usepackage{tikz}
\usepackage{tcolorbox}
\tcbuselibrary{skins,breakable}
\usepackage{booktabs}
\usepackage{fontawesome5}
\usepackage{lmodern}
\usepackage{sectsty}
\usepackage{titlesec}

% ============================================
% GEOMETRÍA Y MÁRGENES MODERNOS
% ============================================
\geometry{
    left=2.5cm,
    right=2.5cm,
    top=2.5cm,
    bottom=2.5cm,
    headheight=24pt
}

% ============================================
% COLORES CORPORATIVOS SIPUD
% ============================================
\definecolor{sipudblue}{RGB}{37, 99, 235}
\definecolor{sipudgray}{RGB}{107, 114, 128}
\definecolor{sipudlight}{RGB}{239, 246, 255}
\definecolor{sipudgreen}{RGB}{34, 197, 94}
\definecolor{sipudred}{RGB}{239, 68, 68}
\definecolor{sipudorange}{RGB}{249, 115, 22}
\definecolor{sipudpurple}{RGB}{147, 51, 234}

% ============================================
% CONFIGURACIÓN DE HYPERREF
% ============================================
\hypersetup{
    colorlinks=true,
    linkcolor=sipudblue,
    urlcolor=sipudblue,
    citecolor=sipudblue,
    pdfborder={0 0 0},
    pdfauthor={SIPUD Team},
    pdftitle={SIPUD - Manual de Usuario}
}

% ============================================
% TIPOGRAFÍA Y SECCIONES
% ============================================
\titleformat{\chapter}[display]
  {\normalfont\huge\bfseries\color{sipudblue}}
  {\chaptertitlename\ \thechapter}{20pt}{\Huge}
\titleformat{\section}
  {\normalfont\Large\bfseries\color{sipudblue}}
  {\thesection}{1em}{}
\titleformat{\subsection}
  {\normalfont\large\bfseries\color{sipudgray}}
  {\thesubsection}{1em}{}

\chapterfont{\color{sipudblue}}
\sectionfont{\color{sipudblue}}
\subsectionfont{\color{sipudgray}}

% ============================================
% ENCABEZADOS Y PIES DE PÁGINA
% ============================================
\pagestyle{fancy}
\fancyhf{}
\fancyhead[LE,RO]{\textcolor{sipudblue}{\textbf{SIPUD}}}
\fancyhead[RE,LO]{\leftmark}
\fancyfoot[C]{\thepage}
\renewcommand{\headrulewidth}{0.5pt}
\renewcommand{\headrule}{\hbox to\headwidth{\color{sipudblue}\leaders\hrule height \headrulewidth\hfill}}
\renewcommand{\footrulewidth}{0pt}

% ============================================
% CAJAS PERSONALIZADAS CON ICONOS
% ============================================

% Caja de Consejo (Tips)
\newtcolorbox{tipbox}{
    enhanced,
    breakable,
    colback=sipudlight,
    colframe=sipudblue,
    arc=4mm,
    boxrule=1pt,
    left=8mm,
    overlay={
        \node[anchor=west, font=\Large] at (frame.west) {\textcolor{sipudblue}{\faLightbulb}};
    },
    fonttitle=\bfseries\color{sipudblue},
    title=Consejo
}

% Caja de Información
\newtcolorbox{infobox}{
    enhanced,
    breakable,
    colback=sipudlight,
    colframe=sipudblue,
    arc=4mm,
    boxrule=1pt,
    left=8mm,
    overlay={
        \node[anchor=west, font=\Large] at (frame.west) {\textcolor{sipudblue}{\faInfoCircle}};
    },
    fonttitle=\bfseries\color{sipudblue},
    title=Información
}

% Caja de Advertencia
\newtcolorbox{warningbox}{
    enhanced,
    breakable,
    colback=orange!5,
    colframe=sipudorange,
    arc=4mm,
    boxrule=1pt,
    left=8mm,
    overlay={
        \node[anchor=west, font=\Large] at (frame.west) {\textcolor{sipudorange}{\faExclamationTriangle}};
    },
    fonttitle=\bfseries\color{sipudorange},
    title=Atención
}

% Caja de Éxito
\newtcolorbox{successbox}{
    enhanced,
    breakable,
    colback=green!5,
    colframe=sipudgreen,
    arc=4mm,
    boxrule=1pt,
    left=8mm,
    overlay={
        \node[anchor=west, font=\Large] at (frame.west) {\textcolor{sipudgreen}{\faCheckCircle}};
    },
    fonttitle=\bfseries\color{sipudgreen},
    title=Completado
}

% Caja de Paso a Paso
\newtcolorbox{stepbox}[1]{
    enhanced,
    breakable,
    colback=sipudpurple!5,
    colframe=sipudpurple,
    arc=4mm,
    boxrule=1pt,
    fonttitle=\bfseries\color{sipudpurple},
    title=#1
}

% ============================================
% CONFIGURACIÓN DE LISTAS
% ============================================
\setlist[itemize]{leftmargin=*,labelsep=5mm}
\setlist[enumerate]{leftmargin=*,labelsep=5mm}

% ============================================
% INICIO DEL DOCUMENTO
% ============================================
\begin{document}

% ============================================
% PORTADA MODERNA
% ============================================
\begin{titlepage}
    \centering
    \vspace*{1cm}
    
    % Logo minimalista con TikZ
    \begin{tikzpicture}
        \shade[ball color=sipudblue!80] (0,0) circle (2cm);
        \node[white, font=\fontsize{40}{48}\selectfont\bfseries] at (0,0.2) {S};
        \draw[sipudblue, line width=4pt] (-2.5,-2.5) -- (2.5,-2.5);
    \end{tikzpicture}
    
    \vspace{1.5cm}
    
    {\fontsize{48}{58}\selectfont\bfseries\color{sipudblue} SIPUD}\\[0.5cm]
    {\Large\color{sipudgray} Sistema Integrado de Gestión Empresarial}\\[2cm]
    
    {\Huge\bfseries Manual de Usuario}\\[0.3cm]
    {\large Guía Rápida y Práctica}\\[3cm]
    
    \begin{tikzpicture}
        \fill[sipudlight, rounded corners=5mm] (0,0) rectangle (12,3);
        \node[sipudblue, font=\large] at (6,2.2) {\faBoxes\ \ Inventario};
        \node[sipudblue, font=\large] at (6,1.5) {\faShoppingCart\ \ Ventas};
        \node[sipudblue, font=\large] at (6,0.8) {\faWarehouse\ \ Bodega};
    \end{tikzpicture}
    
    \vfill
    
    {\large Versión 1.0 • Enero 2026}
    
\end{titlepage}

% ============================================
% ÍNDICE
% ============================================
\tableofcontents
\clearpage

% ============================================
% CAPÍTULO 1: BIENVENIDA
% ============================================
\chapter{Bienvenido a SIPUD}

\section{¿Qué es SIPUD?}

SIPUD es tu solución completa para gestionar productos, ventas y bodega. Todo en un solo lugar, fácil de usar y accesible desde cualquier dispositivo.

\begin{infobox}
\textbf{SIPUD te ayuda a:}
\begin{itemize}
    \item Controlar tu inventario en tiempo real
    \item Gestionar ventas con o sin despacho
    \item Recibir y organizar mercancía
    \item Exportar reportes a Excel
    \item Trabajar con múltiples empresas (multi-tenant)
\end{itemize}
\end{infobox}

\section{Módulos Principales}

\subsection*{\faChartLine\ Dashboard}
Panel de control con métricas clave de tu negocio.

\subsection*{\faBoxes\ Productos}
Catálogo completo con precios, stock y bundles (packs).

\subsection*{\faShoppingCart\ Ventas}
Registra ventas en local o con despacho. Control de pagos y entregas.

\subsection*{\faWarehouse\ Bodega}
Gestión de pedidos, recepción, mermas y vencimientos.

\subsection*{\faChartBar\ Reportes}
Exporta datos a Excel para análisis personalizado.

\subsection*{\faUserCog\ Administración}
Gestión de usuarios y permisos (solo administradores).

\section{Requisitos}

\begin{table}[h]
\centering
\begin{tabular}{@{}ll@{}}
\toprule
\textbf{Componente} & \textbf{Requisito} \\
\midrule
Navegador & Chrome, Firefox, Safari o Edge (versiones recientes) \\
Conexión & Internet estable \\
Resolución & Mínimo 1024×768 \\
\bottomrule
\end{tabular}
\end{table}

\begin{tipbox}
SIPUD funciona en computadores, tablets y celulares. La interfaz se adapta automáticamente al tamaño de tu pantalla.
\end{tipbox}

% ============================================
% CAPÍTULO 2: PRIMEROS PASOS
% ============================================
\chapter{Primeros Pasos}

\section{Iniciar Sesión}

\begin{stepbox}{Paso 1: Acceder al sistema}
\begin{enumerate}
    \item Abre tu navegador
    \item Ingresa a: \texttt{http://[servidor]:5006}
    \item Ingresa usuario y contraseña
    \item Haz clic en \textbf{Iniciar Sesión}
\end{enumerate}
\end{stepbox}

\begin{tipbox}
Si olvidaste tu contraseña, haz clic en \textit{¿Olvidaste tu contraseña?} y sigue las instrucciones por email.
\end{tipbox}

\section{Seleccionar Empresa (Tenant)}

Si trabajas con varias empresas, selecciona la correcta después del login.

\begin{warningbox}
Los datos de cada empresa están separados. Verifica que estés en la empresa correcta antes de trabajar.
\end{warningbox}

\section{Navegar por el Sistema}

\begin{itemize}
    \item \textbf{Menú lateral izquierdo:} Acceso a todos los módulos
    \item \textbf{Barra superior:} Cambiar empresa, notificaciones, tu perfil
    \item \textbf{Dashboard central:} Métricas y acceso rápido
\end{itemize}

\section{Cerrar Sesión}

Haz clic en tu nombre (esquina superior derecha) → \textbf{Cerrar Sesión}.

\begin{infobox}
Por seguridad, el sistema cierra tu sesión automáticamente después de 30 minutos sin actividad.
\end{infobox}

% ============================================
% CAPÍTULO 3: DASHBOARD
% ============================================
\chapter{Dashboard}

El Dashboard es tu panel de control. Aquí ves el estado general de tu negocio:

\begin{itemize}
    \item \textbf{Total Productos:} Cuántos productos tienes en catálogo
    \item \textbf{Ventas del Mes:} Número de ventas del mes actual
    \item \textbf{Stock Crítico:} Productos con poco stock
    \item \textbf{Pedidos Pendientes:} Órdenes de compra sin recibir
\end{itemize}

\begin{tipbox}
Haz clic en cualquier métrica para ver más detalles.
\end{tipbox}

% ============================================
% CAPÍTULO 4: PRODUCTOS
% ============================================
\chapter{Productos}

\section{Ver Catálogo}

En el menú lateral: \textbf{Productos} → verás la tabla con todos tus productos.

\subsection*{Códigos de Color del Stock}
\begin{itemize}
    \item \textcolor{sipudgreen}{\textbf{Verde:}} Stock normal
    \item \textcolor{sipudorange}{\textbf{Naranja:}} Stock bajo
    \item \textcolor{sipudred}{\textbf{Rojo:}} Stock crítico o agotado
\end{itemize}

\section{Crear Producto}

\begin{stepbox}{Agregar un nuevo producto}
\begin{enumerate}
    \item Haz clic en \textbf{Nuevo Producto}
    \item Completa:
    \begin{itemize}
        \item \textbf{Nombre:} Nombre del producto
        \item \textbf{SKU:} Código único (opcional, se genera automático)
        \item \textbf{Categoría:} Tipo de producto
        \item \textbf{Precio Base:} Precio de venta
        \item \textbf{Stock Crítico:} Nivel mínimo de alerta
    \end{itemize}
    \item Haz clic en \textbf{Guardar Producto}
\end{enumerate}
\end{stepbox}

\section{Bundles o Packs}

Un \textbf{bundle} es un pack que contiene otros productos. Por ejemplo: una caja de 12 galletas.

\begin{tipbox}
El stock de bundles se calcula automáticamente según sus componentes. Para tener bundles disponibles, debes \textit{ensamblarlos} en el módulo de Bodega.
\end{tipbox}

\subsection*{Crear un Bundle}
\begin{enumerate}
    \item Crea el producto bundle normalmente
    \item Edítalo y agrega los \textbf{Componentes del Bundle}
    \item Indica la cantidad de cada componente
    \item Guarda los cambios
\end{enumerate}

\section{Editar y Eliminar}

\begin{itemize}
    \item \textbf{Editar:} Haz clic en \faEdit\ junto al producto
    \item \textbf{Eliminar:} Haz clic en \faTrash\ (solo si no tiene stock ni ventas)
\end{itemize}

\begin{warningbox}
No puedes eliminar productos que tengan stock, ventas o pedidos asociados. Primero resuelve esas dependencias.
\end{warningbox}

% ============================================
% CAPÍTULO 5: VENTAS
% ============================================
\chapter{Ventas}

\section{Tipos de Venta}

SIPUD maneja dos modalidades:

\begin{enumerate}
    \item \textbf{Con Despacho:} Entregas a domicilio con seguimiento
    \item \textbf{En Local:} Venta en mostrador con entrega inmediata
\end{enumerate}

\section{Crear Venta con Despacho}

\begin{stepbox}{Nueva venta con entrega}
\begin{enumerate}
    \item Menú → \textbf{Ventas} → \textbf{Nueva Venta}
    \item Completa datos del cliente:
    \begin{itemize}
        \item Nombre completo
        \item Dirección de entrega
        \item Teléfono
        \item Tipo: \textit{Con Despacho}
    \end{itemize}
    \item Agrega productos:
    \begin{itemize}
        \item Busca el producto
        \item Ingresa cantidad
        \item Clic en \textbf{Agregar al Carrito}
    \end{itemize}
    \item (Opcional) Registra pago inicial:
    \begin{itemize}
        \item Marca \textit{Registrar Pago Inicial}
        \item Monto y vía de pago
    \end{itemize}
    \item Clic en \textbf{Confirmar y Crear Venta}
\end{enumerate}
\end{stepbox}

\begin{tipbox}
Puedes dejar el pago pendiente y registrar abonos después desde el botón \textbf{Actualizar}.
\end{tipbox}

\section{Crear Venta en Local}

Igual que la anterior, pero:
\begin{itemize}
    \item Selecciona tipo: \textit{En Local}
    \item No necesitas ingresar dirección
    \item La venta se marca automáticamente como \textit{Entregada}
\end{itemize}

\section{Estados de Entrega}

Las ventas con despacho pasan por estos estados:

\begin{table}[h]
\centering
\begin{tabular}{@{}ll@{}}
\toprule
\textbf{Estado} & \textbf{Descripción} \\
\midrule
Pendiente & Venta registrada, sin preparar \\
En Preparación & Empacando productos \\
En Tránsito & Pedido en camino \\
Entregado & Entrega exitosa \\
Con Observaciones & Entregado con incidencias \\
Cancelado & Venta anulada \\
\bottomrule
\end{tabular}
\end{table}

\section{Actualizar Estado}

\begin{stepbox}{Cambiar estado de una venta}
\begin{enumerate}
    \item En la tabla de ventas, clic en \textbf{Actualizar}
    \item Pestaña \textit{Estado de Entrega}
    \item Selecciona el nuevo estado
    \item (Opcional) Agrega observaciones
    \item Clic en \textbf{Guardar Cambios}
\end{enumerate}
\end{stepbox}

\begin{infobox}
Cuando marcas una venta como \textit{Entregado}, el sistema registra automáticamente la fecha y hora.
\end{infobox}

\section{Gestión de Pagos}

\subsection*{Estados de Pago}
\begin{itemize}
    \item \textcolor{sipudred}{\textbf{Pendiente:}} No hay pagos registrados
    \item \textcolor{sipudorange}{\textbf{Parcial:}} Pago incompleto
    \item \textcolor{sipudgreen}{\textbf{Pagado:}} Total pagado
\end{itemize}

\subsection*{Registrar Abono}
\begin{enumerate}
    \item Clic en \textbf{Actualizar} en la venta
    \item Pestaña \textit{Pagos}
    \item Sección \textit{Registrar Nuevo Pago}:
    \begin{itemize}
        \item Monto
        \item Vía de pago (efectivo, transferencia, etc.)
        \item Referencia (opcional)
    \end{itemize}
    \item Clic en \textbf{Registrar Pago}
\end{enumerate}

\begin{tipbox}
Puedes registrar múltiples abonos hasta completar el total. Cada pago queda en el historial para auditoría.
\end{tipbox}

\section{Exportar Ventas}

Clic en \textbf{Exportar a Excel} para descargar todas las ventas con detalles completos.

% ============================================
% CAPÍTULO 6: BODEGA
% ============================================
\chapter{Bodega}

El módulo de Bodega gestiona todo el flujo de entrada de inventario.

\section{Submódulos}

\begin{enumerate}
    \item \textbf{Pedidos a Proveedores:} Órdenes de compra
    \item \textbf{Recepción de Mercancía:} Entrada de productos
    \item \textbf{Registro de Mermas:} Pérdidas de inventario
    \item \textbf{Gestión de Vencimientos:} Monitoreo de fechas
\end{enumerate}

\section{Pedidos a Proveedores}

\begin{stepbox}{Crear orden de compra}
\begin{enumerate}
    \item Menú → \textbf{Bodega} → \textbf{Pedidos a Proveedores}
    \item Clic en \textbf{Nuevo Pedido}
    \item Completa:
    \begin{itemize}
        \item Proveedor (o crea uno nuevo)
        \item Número de factura
        \item Monto total
        \item Observaciones
    \end{itemize}
    \item Clic en \textbf{Crear Pedido}
\end{enumerate}
\end{stepbox}

El pedido queda con estado \textit{Pendiente}, listo para recepcionar mercancía.

\section{Recepción de Mercancía}

Este es el paso clave para actualizar tu inventario.

\begin{stepbox}{Recepcionar productos}
\begin{enumerate}
    \item \textbf{Recepción de Mercancía} → selecciona pedido pendiente
    \item Clic en \textbf{Recepcionar}
    \item Para cada producto recibido:
    \begin{itemize}
        \item Selecciona producto
        \item Ingresa cantidad
        \item Código de lote (opcional, se autogenera)
        \item \textbf{Fecha de vencimiento}
    \end{itemize}
    \item Agrega todos los productos del pedido
    \item Clic en \textbf{Confirmar Recepción}
\end{enumerate}
\end{stepbox}

\begin{successbox}
El stock se actualiza inmediatamente y los productos quedan disponibles para venta.
\end{successbox}

\subsection*{Sistema de Lotes}

Cada recepción crea \textbf{lotes} independientes con su propia fecha de vencimiento. Esto permite:
\begin{itemize}
    \item Trazabilidad completa
    \item Control automático FIFO
    \item Gestión precisa de vencimientos
\end{itemize}

\begin{infobox}
\textbf{FIFO} (First In, First Out): El sistema descuenta stock siempre del lote más antiguo primero, reduciendo el riesgo de vencimientos.
\end{infobox}

\section{Registro de Mermas}

Las mermas son pérdidas de inventario (vencidos, dañados, robados, etc.).

\begin{stepbox}{Registrar una merma}
\begin{enumerate}
    \item \textbf{Registro de Mermas} → \textbf{Nueva Merma}
    \item Completa:
    \begin{itemize}
        \item Producto afectado
        \item Cantidad perdida
        \item Razón (vencido, dañado, perdido, robo, otro)
        \item Observaciones
    \end{itemize}
    \item Clic en \textbf{Registrar Merma}
\end{enumerate}
\end{stepbox}

El sistema descuenta automáticamente el stock usando FIFO (del lote más antiguo primero).

\begin{warningbox}
No puedes registrar una merma mayor al stock disponible. El sistema te mostrará un error.
\end{warningbox}

\section{Gestión de Vencimientos}

Monitorea las fechas de vencimiento de tus productos.

\subsection*{Códigos de Color}
\begin{itemize}
    \item \textcolor{sipudred}{\textbf{Rojo:}} Vence en 7 días o menos (¡urgente!)
    \item \textcolor{sipudorange}{\textbf{Naranja:}} Vence entre 8-30 días
    \item \textcolor{sipudgreen}{\textbf{Verde:}} Más de 30 días
\end{itemize}

\begin{tipbox}
Para productos próximos a vencer (rojos/naranjas):
\begin{itemize}
    \item Crea ofertas para acelerar la venta
    \item Considera donaciones
    \item Si ya venció, regístralo como merma
\end{itemize}
\end{tipbox}

\section{Ensamblado de Bundles}

Si usas bundles (packs), debes ensamblarlos aquí.

\begin{stepbox}{Ensamblar bundles}
\begin{enumerate}
    \item \textbf{Ensamblado de Bundles}
    \item Selecciona el bundle
    \item Cantidad a ensamblar
    \item El sistema verifica que haya stock de componentes
    \item Clic en \textbf{Ensamblar}
\end{enumerate}
\end{stepbox}

El sistema descuenta componentes (FIFO) y crea stock del bundle ensamblado.

% ============================================
% CAPÍTULO 7: ADMINISTRACIÓN
% ============================================
\chapter{Administración}

\begin{warningbox}
Este módulo solo está disponible para usuarios con rol \textbf{Administrador}.
\end{warningbox}

\section{Gestión de Usuarios}

\subsection*{Crear Usuario}
\begin{enumerate}
    \item \textbf{Administración} → \textbf{Usuarios} → \textbf{Nuevo Usuario}
    \item Completa: nombre de usuario, email, contraseña, rol
    \item Clic en \textbf{Crear Usuario}
\end{enumerate}

\subsection*{Roles Disponibles}

\begin{table}[h]
\centering
\small
\begin{tabular}{@{}ll@{}}
\toprule
\textbf{Rol} & \textbf{Acceso} \\
\midrule
Admin & Todo (incluyendo administración) \\
Manager & Productos, ventas, bodega, reportes \\
Warehouse & Solo bodega y consulta de productos \\
Sales & Solo ventas y consulta de productos \\
\bottomrule
\end{tabular}
\end{table}

\subsection*{Desactivar Usuario}

En lugar de eliminar, es mejor \textit{desactivar}:
\begin{itemize}
    \item Edita el usuario
    \item Marca \textit{Usuario Inactivo}
    \item El usuario no podrá iniciar sesión
    \item Sus registros históricos se mantienen
\end{itemize}

\section{Registro de Actividades}

Historial completo de todas las operaciones del sistema:
\begin{itemize}
    \item Quién hizo qué y cuándo
    \item IP y navegador utilizado
    \item Útil para auditoría y seguridad
\end{itemize}

\begin{tipbox}
Usa los filtros (usuario, módulo, acción, fecha) para encontrar actividades específicas.
\end{tipbox}

% ============================================
% CAPÍTULO 8: REPORTES
% ============================================
\chapter{Reportes}

\section{Exportar Ventas}

\begin{enumerate}
    \item \textbf{Reportes} → \textbf{Ventas}
    \item (Opcional) Aplica filtros: fechas, estados, cliente
    \item Clic en \textbf{Exportar a Excel}
\end{enumerate}

El archivo incluye: ID, fecha, cliente, dirección, total, estado entrega, estado pago, etc.

\section{Exportar Inventario}

\subsection*{Stock Actual}
Productos con stock disponible, stock crítico y estado.

\subsection*{Lotes Detallados}
Cada lote individual con fecha de vencimiento, cantidad y proveedor.

\begin{tipbox}
Los archivos Excel están optimizados para usar con \textbf{tablas dinámicas}. Así puedes crear análisis personalizados sin modificar SIPUD.
\end{tipbox}

% ============================================
% CAPÍTULO 9: PREGUNTAS FRECUENTES
% ============================================
\chapter{Preguntas Frecuentes}

\section{Acceso}

\textbf{¿Olvidé mi contraseña, qué hago?}

En login → \textit{¿Olvidaste tu contraseña?} → ingresa tu email → sigue las instrucciones.

\medskip

\textbf{¿Puedo usar SIPUD desde mi celular?}

Sí, SIPUD es responsive y funciona en cualquier dispositivo.

\section{Productos}

\textbf{¿Por qué no puedo eliminar un producto?}

Solo puedes eliminar productos que NO tengan:
\begin{itemize}
    \item Stock en bodega
    \item Ventas asociadas
    \item Pedidos pendientes
\end{itemize}

\medskip

\textbf{¿Cómo funciona el stock de bundles?}

Se calcula automáticamente según componentes disponibles. Para tener bundles, debes ensamblarlos en Bodega.

\section{Ventas}

\textbf{¿Puedo modificar una venta después de crearla?}

Puedes actualizar: estado de entrega, agregar pagos y modificar observaciones. No puedes cambiar productos ni cantidades.

\medskip

\textbf{¿Puedo registrar abonos parciales?}

Sí, SIPUD soporta múltiples pagos hasta completar el total.

\section{Bodega}

\textbf{¿Qué es FIFO?}

First In, First Out: el sistema descuenta stock siempre del lote más antiguo primero.

\medskip

\textbf{¿Puedo tener varios lotes del mismo producto?}

Sí, cada recepción crea un lote separado. El sistema los gestiona automáticamente.

\section{Problemas Comunes}

\begin{table}[h]
\centering
\small
\begin{tabular}{@{}p{5cm}p{6cm}@{}}
\toprule
\textbf{Error} & \textbf{Solución} \\
\midrule
Stock insuficiente & Recepciona más mercancía \\
Sesión expirada & Vuelve a iniciar sesión \\
Fecha de vencimiento inválida & No puede estar en el pasado \\
Sin permisos & Contacta al administrador \\
\bottomrule
\end{tabular}
\end{table}

\begin{infobox}
\textbf{¿Necesitas más ayuda?}

Contacta a:
\begin{itemize}
    \item Tu administrador del sistema
    \item Soporte técnico: \texttt{soporte@sipud.com}
\end{itemize}

Incluye: descripción del problema, pasos para reproducirlo y capturas de pantalla.
\end{infobox}

% ============================================
% APÉNDICE: GLOSARIO
% ============================================
\chapter*{Glosario}
\addcontentsline{toc}{chapter}{Glosario}

\begin{description}[labelwidth=3cm,leftmargin=3.5cm]
    \item[Bundle] Pack o caja compuesta por otros productos.
    
    \item[FIFO] First In, First Out. Método donde se usan primero los productos más antiguos.
    
    \item[Lote] Conjunto de productos con misma fecha de vencimiento.
    
    \item[Merma] Pérdida de inventario por vencimiento, daño, robo, etc.
    
    \item[Multi-tenant] Sistema que gestiona múltiples empresas separadamente.
    
    \item[SKU] Código único de producto.
    
    \item[Stock Crítico] Nivel mínimo de alerta de inventario.
    
    \item[Tenant] Empresa u organización en el sistema.
\end{description}

\vfill

\begin{center}
\begin{tcolorbox}[width=0.8\textwidth,colback=sipudlight,colframe=sipudblue,arc=5mm]
\centering
\Large\textbf{¡Gracias por usar SIPUD!}\\[0.3cm]
\normalsize Tu socio en la gestión empresarial
\end{tcolorbox}
\end{center}

\end{document}
