\documentclass[12pt]{article}
\usepackage[utf8]{inputenc}
\usepackage[spanish]{babel}
\usepackage{geometry}
\usepackage{enumitem}
\usepackage{titlesec}
\usepackage{hyperref}
\usepackage{xcolor}
\usepackage{amsmath}

% Configuración de página
\geometry{a4paper, margin=2.5cm}
\titleformat{\section}{\normalfont\Large\bfseries\color{darkblue}}{}{0pt}{}
\definecolor{darkblue}{rgb}{0.0, 0.0, 0.5}

\title{
    \textbf{Planificación Técnica de Desarrollo}\\
    \large Sistema de Gestión de Inventario y Cuadratura Comercial
}
\author{Consultoría de Software - Flask/SQLite Stack}
\date{12 de enero de 2026}

\begin{document}

\maketitle

\section{Introducción}
Este documento detalla la planificación para la migración de un sistema basado en hojas de cálculo hacia una aplicación web centralizada utilizando el micro-framework \textbf{Flask} y \textbf{SQLite}. El objetivo es eliminar el desorden de datos y automatizar la fórmula: 
\[ \text{Stock Actualizado} = \text{Stock Inicial} + \text{Compras} - \text{Ventas} \]

\section{Módulos Críticos y Flujo de Datos}

\subsection{1. Módulo de Abastecimiento y Lotes}
Basado en la hoja ``Detalle Stock'', se implementará un sistema de trazabilidad por lotes.
\begin{itemize}
    \item \textbf{Gestión de Lote PD (Product Data):} Registro de fecha de recepción, elaboración y vencimiento.
    \item \textbf{Control de Proveedor:} Almacenamiento de Código de Proveedor y N° de Factura de entrada.
    \item \textbf{Estado de Facturación:} Marcado de facturas como ``Pagado'' o ``Pendiente'' para el flujo de caja.
\end{itemize}

\subsection{2. Módulo de Logística y Reparto}
Este módulo digitaliza la ``Hoja de Repartidor'', optimizando el seguimiento físico:
\begin{itemize}
    \item \textbf{KPIs de Ruta:} Registro de Kilometraje (Km) y tiempo de ruta para medir eficiencia.
    \item \textbf{Geo-referenciación:} Campos de Sector y Ciudad para futuros mapas de calor de ventas.
    \item \textbf{Vínculo de Venta:} Cada salida de bodega debe estar amarrada a una Nota de Venta o Factura.
\end{itemize}

\subsection{3. Motor de Cuadratura}
Es el corazón del sistema. Debe comparar las ventas de plataformas externas (como Jumpseller) con el inventario físico disponible.
\begin{itemize}
    \item \textbf{Conciliación:} Validación automática de montos y cantidades facturadas vs. despachadas.
\end{itemize}

\subsection{4. Módulo de Armado de Cajas (Kitting)}
Funcionalidad crítica para evitar la duplicidad de inventario al manejar productos compuestos (Cajas o Packs).
\begin{itemize}
    \item \textbf{Transformación de Stock:} Al "armar" una caja, el sistema descuenta automáticamente el stock de los componentes individuales y aumenta el stock de la caja terminada.
    \item \textbf{Consistencia:} Se valida que existan suficientes componentes antes del armado, asegurando que el total de stock valorizado se mantenga constante.
\end{itemize}

\section{Especificaciones Técnicas}

\subsection{Modelado de Base de Datos (SQLAlchemy)}
Se utilizarán las siguientes tablas principales:
\begin{enumerate}[label=\alph*)]
    \item \textbf{User:} Control de acceso para el personal.
    \item \textbf{Product:} Maestro de artículos con stock crítico.
    \item \textbf{LotDetail:} Almacén de fechas de vencimiento y códigos específicos.
    \item \textbf{Sale:} Registro de ventas (integrando campos de e-commerce).
    \item \textbf{LogisticsRoute:} Registro de despachos y métricas de transporte.
\end{enumerate}



\subsection{Interfaz de Usuario (UX/UI)}
\begin{itemize}
    \item \textbf{Estilo:} Minimalista basado en Tailwind CSS, priorizando la legibilidad en dispositivos móviles para el personal de bodega y repartidores.
    \item \textbf{Responsividad:} Tablas con scroll horizontal para datos extensos de stock y botones de acción rápida para entrada/salida.
\end{itemize}

\section{Cronograma de Implementación}
\begin{description}
    \item[Semana 1:] Configuración del entorno Flask y migración inicial (ETL) de los Excels actuales a SQLite.
    \item[Semana 2:] Desarrollo de formularios de ingreso (Abastecimiento) y lógica de lotes.
    \item[Semana 3:] Implementación de vistas de logística y reportes de cuadratura.
    \item[Semana 4:] Despliegue en servidor (Render/Railway) y capacitación de usuario final.
\end{description}

\section{Implementación Técnica Realizada}

\subsection{Esquema de Base de Datos}
Se ha implementado el siguiente modelo relacional en SQLite:

\begin{itemize}
    \item \textbf{User:} Gestión de usuarios con roles (admin, bodega, driver).
    \item \textbf{Supplier:} Proveedores de productos.
    \item \textbf{Product:} Catálogo maestro (SKU, Nombre, Stock Crítico). Relacionado 1 a N con \texttt{Lot}. Identifica si es "Bundle" (Caja).
    \item \textbf{ProductBundle:} Tabla intermedia que define la "receta" de una Caja (qué componentes y en qué cantidad).
    \item \textbf{InboundOrder:} Cabecera de recepción de productos (Facturas de compra).
    \item \textbf{Lot:} Trazabilidad de lotes. Incluye \texttt{lot\_code}, fecha de vencimiento y cantidades.
    \item \textbf{Sale:} Cabecera de pedido/venta. Estados: pending, assigned, in\_transit, delivered.
    \item \textbf{SaleItem:} Detalle de productos por venta. Normaliza la estructura plana de los Excel antiguos.
    \item \textbf{LogisticsRoute:} Agrupación de ventas para despacho asignadas a un conductor.
\end{itemize}

\subsection{Manual de Operación (Walkthrough)}

\subsubsection{Ejecución del Entorno}
Para iniciar el sistema en entorno local:

\begin{enumerate}
    \item Activar el entorno virtual:
    \begin{verbatim}
    source venv/bin/activate
    \end{verbatim}
    \item Ejecutar la aplicación Flask:
    \begin{verbatim}
    export FLASK_APP=run.py
    flask run
    \end{verbatim}
\end{enumerate}
La aplicación estará disponible en \texttt{http://127.0.0.1:5000/}.

\subsubsection{Gestión de Base de Datos}
La base de datos se encuentra en \texttt{instance/inventory.db}. Para actualizar el esquema ante nuevos cambios:
\begin{verbatim}
flask db migrate -m "Descripción"
flask db upgrade
\end{verbatim}

\end{document}